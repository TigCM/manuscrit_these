\chapter{Les atomes de Rydberg circulaires en interaction : vers un simulateur quantique}
\label{chapter:circsim}

Intro : pourquoi envisager les Rydberg circulaires comme plateforme de simulation ?

\section{Les interactions dipôle-dipôle entre atomes de Rydberg circulaires}

\section{Le hamiltonien $XXZ$ simulé}
	\subsection{Mise sous forme XXZ}
	\subsection{"Tunabilité" des interactions}
		Comment justifier le domaine de champs électrique et magnétique statiques dans lequel on se place, avant d'avoir parlé du problème de la durée de vie ??
	\subsection{Quelle physique ? Diagramme de phase}

\section{Principes techniques du simulateur}
	\subsection{Piégeage laser des Rydberg circulaires}
	\subsection{Préservation des états de Rydberg}
		\subsubsection*{Temps de vie dans l'espace libre}
		\subsubsection*{Inhibition de l'émission spontanée}
		\subsubsection*{Problème du mixing et solution}
	\subsection{Préparation déterministe d'une chaîne}

\newpage
Reprendre le PRX