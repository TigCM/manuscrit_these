\chapter{Dispositif expérimental : des atomes de Rydberg sur puce}
\label{chapter:setup_ryd}

\section{Excitation et détection d'atomes de Rydberg}
	\subsection*{Schémas d'excitation}
		\noindent schéma laser : schéma de niveaux (60s ou 50d) et schéma optique		
	\subsection*{Schémas de détection}
		\noindent state selective ionization
		\noindent signaux d'ionisation et toutes les subtilités
		
\section{Problème des champs électriques près d'une puce}
	\subsection*{L'élargissement Stark inhomogène}
		\noindent raies de plusieurs dizaines de MHz de large, drift
	\subsection*{Recouvrir la puce de rubidium}
		\noindent dispositif dispensers et raies fines
	\subsection*{Contrôle du champ électrique}
		\noindent Lhomond et CdF, électronique de contrôle \\
		\noindent électrodes RF pour la circularisation (Simion ?) 
		
\section{Spectroscopie microonde}
	\noindent mention rapide des domaines de transition entre les niveaux de Rydberg