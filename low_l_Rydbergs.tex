\chapter{Interaction entre atomes de Rydberg sphériques}
\label{chapter:60s}
%Excitation optique d'atomes de Rydberg à bas $l$ et simulations

\section{Régimes d'excitation en nuage dense : blocage et facilitation}
	\subsection*{Rappels sur l'interaction}
		\noindent hamiltonien d'interaction de paire \\
		approximation de $N$ atomes comme somme de $\frac{N(N-1)}{2}$ paires
	\subsection*{Le blocage dipolaire et la facilitation}
		\noindent les deux régimes d'excitation déterminée par les interactions \\
		négligeabilité des excitations de paires		

\section{Spectroscopie optique du nuage et simulations}
	\subsection*{Élargissement de la raie laser par interactions}
		\noindent conséquence de la facilitation
	\subsection*{Simulations}
		\noindent modèle d'équation de taux\\
		\noindent résultats de simulations comparés aux manips\\
		\noindent question du chauffage
		
\section{Spectroscopie microonde du nuage}
	\subsection*{Spectre des énergies d'interaction du nuage}
		\noindent détails sur la spectro 60s-57s, dont la quasi absence de terme d'échange dans l'interaction
	\subsection*{Mouvement du nuage de Rydbergs}
		\noindent Le gaz gelé ne marche pas !