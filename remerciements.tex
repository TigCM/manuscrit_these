\chapter*{Remerciements}

La présentation de mon travail de thèse dans ce manuscrit ne fait que concentrer dans un bel écrin les aspects scientifiques des quatre années qui ont été nécessaires à sa réalisation.
Ces années ont été marquées de joies et de peines, de vie et d'émotions, à la fois au laboratoire et en-dehors.
Évidemment, tout cela est très lié aux personnes que je tiens à remercier ici.

\bigskip
En premier lieu et bien que l'on aime pas les titres hiérarchiques, je tiens à remercier les \og chefs \fg{} de l'équipe d'Électrodynamique Quantique en Cavité, Michel Brune, Jean-Michel Raimond et Serge Haroche.
Ayant commencé mon stage et ma thèse après que Serge a reçu son prix Nobel, je n'ai pas pu profiter de son encadrement.
Je le remercie néanmoins de m'avoir accueilli dans son équipe, croyant savoir l'intérêt qu'il n'a pas cessé de porter à l'avancée de nos travaux et en reconnaissance de l'environnement scientifique exceptionnel qu'il a su insuffler à l'équipe.
Ce sont Michel et Jean-Michel qui m'ont directement recruté en décembre 2012, alors que j'étais étudiant en Master 2, sans autre qualité remarquable.
Je tiens donc à les remercier de la confiance qu'ils m'ont accordée en décidant de m'accueillir dans l'équipe, confiance qui ne s'est jamais démentie même dans les moments les plus décourageants du déménagement et de la reconstruction de l'expérience au Collège de France.
C'est à travers eux que j'ai appris à \og maniper \fg{} et à réfléchir efficacement aux problèmes qui se présentaient, à faire de la physique expérimentale en somme, des secrets des jonctions d'indium à l'étude des interactions entre atomes de Rydberg, en passant par les bouchons de Stycast et la programmation de simulations numériques.
J'essaierai dans mes recherches futures, de combiner le flegme et la persévérance de Michel avec les fermes injonctions de Jean-Michel envers les appareils qui refusent de marcher ou les atomes récalcitrants.

\bigskip
L'équipe CQED ne serait rien non plus sans ses jeunes membres permanents, Clément, Sébastien et Igor.
Au début de ma thèse, Sébastien a suivi et épaulé les travaux de R14 en plus des deux projets expérimentaux qu'il menait lui-même.
Son aide lors des manips de 2013-2014 a été inestimable.
L'importance de son travail de gestion de l'hélium dès notre arrivée au Collège de France se fait cruellement sentir dès qu'il s'absente.
Je tiens à le remercier pour son aide au labo, pour sa disponibilité et son soutien.
Clément est arrivé un an après le déménagement en septembre 2015, pendant les heures sombres de la nouvelle salle E20.
Il m'a permis de retrouver, petit à petit, la motivation et le plaisir de persévérer dans les efforts, sur une manip difficile et un peu caractérielle.
Son énergie et sa détermination ont été précieuses à tous les niveaux, de même que ses indéniables qualités de physicien expérimentateur qui ont été indispensable à la suite de notre travail une fois que la manip a recommencé à fonctionner.
Je le remercie pour tout cela et pour la pertinence de ses remarques sur les communications orales que j'ai préparées et sur mon manuscrit de thèse.
Igor a toujours été disponible pour m'aider à déboguer les programmes de contrôle de l'expérience, à les développer et les améliorer.
Je le remercie pour cette contribution, indispensable et si souvent ingrate, à mon travail.

\bigskip
J'ai eu la chance de commencer à travailler ici en compagnie de Raul et Carla.
J'en retiendrai la gentillesse, la bienveillance et la volonté de transmission de Raul, qui a représenté pour moi le modèle d'un thésard équilibré et brillant.
Carla m'a offert un soutien moral salvateur au laboratoire alors que nos vies personnelles respectives se sont trouvées concomitamment bouleversées  à l'automne 2013.
J'ai appris d'eux, entre autres techniques, les rudiments du contrôle de la manip, le couplage des lasers dans les fibres optiques, la soudure des boîtiers d'électronique et les trop fameux transferts d'hélium, qui ont rythmé une bonne partie de mes quatre années de thèse, ainsi que le travail en équipe, chose indispensable à la recherche en physique expérimentale.
En relisant les remerciements de la thèse de Raul, je me rappelle la fête d'anniversaire surprise, qu'à son insu et presque contre son gré, nous lui avons organisée avec Carla.
Je les remercie pour ce qu'ils m'ont apporté et pour tous ces souvenirs.
%
Trois des quatre années de ma thèse se sont écoulées en compagnie de Thanh Long.
Malgré quelques moments de friction, je me souviens de son efficacité lors du déménagement et de la reconstruction de la manip, de la joie de nos victoires sur les lasers et les structures en profilé d'aluminium.
Les propositions qu'il a développées à la fin de sa thèse sont centrales dans le futur de la manip et dans l'orientation qu'a prise la fin de la mienne.
Je le remercie pour toutes ces contributions et pour les bons moments que nous avons partagés.
%
En même temps que Clément est arrivé Rodrigo, apportant avec lui sa bonne humeur, son envie d'apprendre et sa curiosité sans limite, sa gentillesse et ses questions toujours renouvelées sur les fondements du monde quantique.
Entre les fabrications de nouvelles puces et le montage du circuit de contrôle RF, il a su percer les secrets du 2D-MOT et des mélasses optiques, reprendre avec vigueur toutes les simulations numériques nécessaires et contribuer significativement au projet de simulateur.
Je ne saurais assez le remercier pour ce qu'il a apporté, à mon travail de thèse bien sûr, mais aussi à ma vie au laboratoire.
%
Enfin, Brice est arrivé en stage au printemps 2017 et a commencé sa thèse avec nous début septembre.
Je n'ai eu que peu l'occasion de travailler avec lui, absorbé que j'étais dans mes dernières manips puis dans la rédaction de mon manuscrit.
Je sais néanmoins qu'il assurera la relève et l'en remercie par avance.

\bigskip
Malgré ma présence plus qu'erratique au déjeuner, j'ai eu le plaisir de côtoyer lors de ma thèse d'autres collègues qui travaillaient ou travaillent encore sur les autres manips de l'équipe. Stefan et les chocolats qu'il ramenait d'Autriche, les fameux \og Mozart balls \fg{} ; Théo et Medhi, et les jams de blues au Caveau des Oubliettes ; les deux Adrien, Brian et Jay, les cool kids avec leur belle manip et leur réussite ; Valentin qui n'a jamais froid ; Arthur et Frédéric et nos trop rares aventures dans les défilés entre République et Nation.
Je garde une mention particulière pour Mariane, dont le soutien et l'amitié m'ont été précieux ; pour Dorian, nos discussions entre deux bureaux ou autour d'un café, nos quelques virées de motards et ses questions sur Origin ; et pour Eva, qui fut une excellente presque-voisine et sur qui je sais pouvoir compter malgré l'espacement de nos moments partagés.
Je les remercie d'avoir été là et je suis heureux d'avoir travaillé à leurs côtés.

\bigskip
D'autres thésard.e.s du LKB ont marqué ces quelques années : Camille, Davide et Andrea, que l'on pourrait appeler la bande des buffets, Chayma, Sébastien, et Marion.
Le travail que j'ai pu accomplir aurait été impossible sans le travail remarquable des services administratifs et techniques du LKB, du Département de Physique et du Collège de France.
Je remercie, pour tout ce qu'ils ont fait et font encore, Audrey, Carmen, Françoise, Binta et Thierry pour l'administration, qui sont d'un soutien indispensable au bon fonctionnement du laboratoire ; Pascal et Nicolas de l'atelier de mécanique pour leur réactivité et les belles pièces qu'ils fabriquent ; Florence Thibout, spécialiste du verre, qui m'a aidé à trouer un miroir, ce qui est plus difficile qu'on ne pourrait l'imaginer ; Olivier, Aurélien et Florian du service de cryogénie de l'ENS qui continuent à nous fournir à distance en hélium liquide.
Je tiens aussi à remercier nos partenaires d'autres laboratoires, qui nous ont aidé avec la fabrication des puces et l'entretien du 2D-MOT, en particulier Christine et David de l'Observatoire de Paris.

%---
%
%Maman, Yves, Coraline
%
%Solène, Lucie et Mélanie
%
%Antoine, Eléonore, Robin & Manuela
%
%La drôme team et les GPPM
%
%ALEX
\bigskip
Bien évidemment, ma thèse s'est aussi faite en-dehors du laboratoire, et je tiens tout autant à remercier les personnes qui m'ont apporté leur confiance, leur soutien, leur amitié et leurs encouragements pendant ces quatre années.
Je commencerai par remercier ma mère, qui m'a aidé à surmonter les périodes difficiles, professionnellement comme personnellement, entre autres à travers la confiance et la fierté qu'elle me témoigne et le soutien moral et matériel qu'elle m'offre depuis toujours.
Merci à Coraline également, ma  s\oe ur qui m'a été aux première loges de mes deux premières années de thèse, qui m'a accueilli tant de fois pour déjeuner et me consoler de mes déboires sentimentaux ou expérimentaux, ou encore se réjouir avec moi des heureux événements, qui m'a souhaité un joyeux anniversaire depuis la fenêtre de son appartement que j'arrivais à discerner depuis la terrasse du Collège de France.
Merci à Yves, à sa confiance bienveillante dans le fait que tout peut être surmonté, à tous ses conseils de mécanicien avisé lorsqu'il m'a fallu, à de si nombreuses reprises, réparer ma moto !
Merci à mon père, dont je sais malgré nos relations erratiques, combien il me soutient et combien il est fier.

\bigskip
Je remercie aussi Antoine \& Éléonore pour toutes ces soirées à refaire le monde, pour le havre de paix, de compréhension et de bien-être qu'est pour moi leur présence et leur appartement puis leur maison de Fleury-lès-Aubrais.
Mes deux autres grand.e.s cousin.e.s, Robin \& Manuela, je remercie pour d'autres soirées à refaire le monde, entre fermes de permaculture et jeux de société.
Je remercie aussi les autres membres de ma famille, Christine, Karin, Anne, Thierry, Nicolas et Élisa, ainsi qu'Olympe la toute dernière, pour ces quelques moments annuels de réjouissances que sont nos réunions de famille, mais aussi pour le soutien et les encouragements de chacun.e.

\bigskip
Dans mon quotidien à Paris, j'ai eu d'autres familles.
Lucie, et Émilien, que je remercie d'avoir toujours été là, pour les innombrables fois où ils m'ont accueilli chez eux, pour tout ce qu'ils m'ont aidé à traverser.
Ma tentative d'occulter leur départ au Mexique en gardant leur valeureuse \og El Fuego II \fg{} ayant échoué, j'ai hâte de pouvoir leur rendre visite dans leur nouvelle vie et suis toujours impatient qu'ils reviennent ne serait-ce que pour quelques jours.
Solène, à qui je n'ai jamais su dire à quel point son amitié est importante pour moi, que je remercie pour sa présence, pour sa franchise et sa clairvoyance et pour tant de pintes de bière partagées aux Marsouins, au Rollins ou au Frog.
Et bien sûr pour avoir été à l'origine de ma rencontre avec Alex !
Mélanie, la meilleure des colocataires, je remercie pour son inébranlable amitié depuis toutes ces années, une amitié qui trop souvent se passe de mots.

\bigskip
Outre ces amitiés individuelles, je tiens également à remercier deux groupes qui m'ont été salutaires pendant ces quatre années.
La \og Drôme team \fg{} : Valérie, Maïté, Émilie, Alice, Antoine, Mikhaël et Ileyk, en souvenir de nos vacances ensemble à La Roche, de nos nombreux réveillons du nouvel an, et de toutes nos aventures parisiennes.
Je les remercie d'avoir été ce groupe de fêtes, de pique-niques et d'encouragements.
La troupe des Gens Pensent Pas Moins : Marion, Brice, Geoffrey, Anna, Florian, Amélie et Nathan, avec qui j'ai eu tant de plaisir à remonter sur les planches.
Reprendre le théâtre avec eux pendant trois ans a été à la fois un grand plaisir et un grand soulagement pour moi.
Je les remercie pour les beaux projets que l'on a montés ensemble, pour l'esprit collectif et libre de la troupe.
Je remercie, enfin, Julien, Cristina, Alice, Robin, Damien, Roxane, pour leurs amitiés et les moments partagés ensemble.

\bigskip
Pour finir, je souhaite remercier Alex.
Je n'imagine pas comment j'aurais pu terminer cette thèse sans elle.
Notre rencontre en 2015 m'a rendu bonheur et joie de vivre.
Elle a toujours su être là pour moi, y compris dans ses périodes intenses entre semaines d'étude et weekends de travail, y compris dans mes périodes intenses de travail expérimental vespéral, jusqu'à la \textit{danger zone} rédactionnelle.
Son amour et son soutien ont été les piliers de mes deux dernières années et seront ceux des années à venir.
Puissé-je lui exprimer son importance et lui offrir un tel appui dans notre commun avenir :\\
\emph{là, tout n'est qu'ordre et beauté, luxe, calme et volupté}.

\vfill
\vspace{1.5em}

\begin{fquote}[Charles Baudelaire][Spleen et Idéal]
\vspace{-1em}
\begin{center}
\hspace{-.0\linewidth}\textsc{Correspondances}
\end{center}

\hspace{.12\linewidth}La Nature est un temple où de vivants piliers \\[-1.5em]

\hspace{.12\linewidth}Laissent parfois sortir de confuses paroles ;\\[-1.5em]

\hspace{.12\linewidth}L'homme y passe à travers des forêts de symboles \\[-1.5em]

\hspace{.12\linewidth}Qui l'observent avec des regards familiers. \bigskip

\hspace{.12\linewidth}Comme de longs échos qui de loin se confondent \\[-1.5em]

\hspace{.12\linewidth}Dans une ténébreuse et profonde unité, \\[-1.5em]

\hspace{.12\linewidth}Vaste comme la nuit et comme la clarté,\\[-1.5em]

\hspace{.12\linewidth}Les parfums, les couleurs et les sons se répondent. \bigskip

\hspace{.12\linewidth}Il est des parfums frais comme des chairs d'enfants, \\[-1.5em]

\hspace{.12\linewidth}Doux comme les hautbois, verts comme les prairies, \\[-1.5em]

\hspace{.12\linewidth}-- Et d'autres, corrompus, riches et triomphants, \bigskip

\hspace{.12\linewidth}Ayant l'expansion des choses infinies, \\[-1.5em]

\hspace{.12\linewidth}Comme l'ambre, le musc, le benjoin et l'encens, \\[-1.5em]

\hspace{.12\linewidth}Qui chantent les transports de l'esprit et des sens.
\end{fquote}