\chapter{Atomes de Rydberg alcalins : des hydrogénoïdes géants}
\label{chapter:Rydberg}
%Atomes de Rydberg : à bas $l$ ou circulaires


\section{Théorie générale des Rydberg}
	\subsection*{Hamiltonien de l'atome d'hydrogène}
		\noindent particularités aux grands $n$
	\subsection*{Défaut quantique : comment passer aux alcalins}
		\noindent le défaut quantique comme un $n$ effectif
		\noindent quantitativement : $\delta_{n,l,j}$ pour les niveaux qui nous concernent
	\subsection*{Les grandes caractéristiques des Rydberg}
		\noindent gigantisme du dipole, sensibilité au champ EM, interactions
		\noindent lois d'échelle


\section{Premier cas particulier : les interactions dipole-dipole et VdW à bas $l$}
	\subsection*{Deux atomes de Rydberg}
		\noindent hamiltonien d'interaction
		\noindent dipole-dipole
		\noindent Van der Waals
		\noindent interaction d'échange
	\subsection*{Blocage dipolaire}
		\noindent blocage et facilitation : preview ?


\section{Second cas particulier : les atomes de Rydberg circulaires}
	\subsection*{L'effet Stark et les Rydberg en champ électrique}
		\noindent Stark maps à grand $l$
	\subsection*{Niveaux paraboliques}
		\noindent échelle des niveaux de la multiplicité en champ \og fort \fg{}
	\subsection*{Caractéristiques des Rydberg circulaires}
		\noindent taille, dipole
		\noindent transitions vers les niveaux proches et émission spontanée, temps de vie
	
	
	
\section{Atomes de Rydberg circulaires en interaction}
	\noindent petit flash forward vers le futur de la manip ?
