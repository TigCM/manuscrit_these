\chapter{Des atomes de Rydberg circulaires sur puce}
\label{chapter:50c}
%Excitation d'atomes de Rydberg circulaires et piégeage laser

\section{Comment exciter des atomes de Rydberg circulaires}
	\subsection*{Les niveaux atomiques du fondamental au Rydberg circulaire}
		\noindent schéma de niveaux et Stark maps
	\subsection*{Spectroscopie 5s-50d}
		\noindent en champ nul et en champ non-nul -> choix de $m_j$
	\subsection*{Spectroscopie 50d-50f}
		\noindent en champ nul et en champ non-nul -> choix de $m_l$ et problème d'élargissement
	\subsection*{Le passage adiabatique}
		\noindent et le dispositif radio-fréquence

\section{Comment caractériser les atomes de Rydberg circulaires}
	\subsection*{Spectroscopie microonde}
		\noindent 50c-51c et optimisation de la RF\\
		\noindent 50c-49c ?
	\subsection*{Temps de vie}
		\noindent temps de vie théorique, temps de vie mesuré, température effective
	\subsection*{Temps de cohérence}
		\noindent franges de Ramsey
	

%\section{Théorie de la force pondéro-motrice appliquée aux 50C} -> goes to chapter:circsim
%	\noindent et description du laser de piégeage

\section{Éjectable : Première évidence du piégeage des atomes circulaires}
	\noindent chute par gravité et/ou expansion du nuage compensée par un tube de lumière
