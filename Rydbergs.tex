\chapter{Atomes de Rydberg alcalins en interaction}
\label{chapter:Rydberg}
%Atomes de Rydberg : à bas $l$ ou circulaires

\section{Théorie générale des Rydberg}
	\subsection*{Hamiltonien de l'atome d'hydrogène}
		\noindent particularités aux grands $n$
	\subsection*{Défaut quantique : comment passer aux alcalins}
		\noindent le défaut quantique comme un $n$ effectif
		\noindent quantitativement : $\delta_{n,l,j}$ pour les niveaux qui nous concernent
	\subsection*{Niveaux à bas $l$}
		\noindent description rapide des spécificités et schéma de niveaux
		\noindent taille, dipole
		\noindent transitions vers niveaux proches, émission spontanée, temps de vie
	\subsection*{Niveaux circulaires}
		\noindent description rapide des spécificités et schéma de niveaux
		\noindent taille, dipole
		\noindent transitions vers niveaux proches, émission spontanée, temps de vie
	\subsection*{Les grandes caractéristiques des Rydberg}
		\noindent gigantisme du dipole, sensibilité au champ EM, interactions
		\noindent lois d'échelle

\section{Atomes de Rydberg en interaction}
	\subsection*{Deux atomes de Rydberg}
		\noindent hamiltonien d'interaction
		\noindent dipole-dipole
		\noindent Van der Waals
		\noindent interaction d'échange
	\subsection*{les interactions entre Rydberg de bas $l$}
		\noindent origine du $C_6$ pour 60s-60s et $C_6/A_6$ avec les voisins
		\noindent blocage dipolaire et facilitation
	\subsection*{les interactions entre Rydberg circulaires}
		\noindent $C_6$ pour 50c-50c et $C_6/A_6$ avec les voisins