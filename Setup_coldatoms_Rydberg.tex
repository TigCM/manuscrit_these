\chapter{Des atomes de Rydberg froids en environnement cryogénique}
\label{chapter:setup_coldatoms_Rydberg}

\section{Les atomes froids}
	\subsection{le cryostat et la puce}
	\subsection{séquence de piégeage et refroidissement}
		\noindent piégeage magnéto-optique
		\noindent piégeage magnétique
		\noindent refroidissement évaporatif jusqu'au BEC
	\subsection{imagerie atomique}
		\noindent optique d'imagerie
		\noindent imagerie par absorption
		\noindent tweaks : absorption no-log et réduction des franges
	\subsection{nuages typiques}
		\noindent quels MOTs, mélasses et nuages froids obtenus

\section{Excitation et détection d'atomes de Rydberg près d'une puce}
	\subsection{schéma d'excitation}
		\noindent laser et niveaux atomiques
	\subsection{schéma de détection}
		\noindent décrire la détection sélective en champ
	\subsection{problème de champs électriques et flash de Rb}
		\noindent vieilles raies larges et moches
		\noindent flash Rb
		\noindent belles raies fines
	\subsection{compensation et contrôle des champs}
		\noindent électrode Vcomp
		\noindent électrodes RF
	\subsection{manipulation et observation des Rydbergs}
		\noindent la spectroscopie microondes