\logos{}{}{}%./logo/logo.pdf}{}{} 
\author{Tigrane Cantat-Moltrecht}
\thesis{THÈSE DE DOCTORAT \protect\\DE L'ÉCOLE NORMALE SUPÉRIEURE}
\specialite{PHYSIQUE QUANTIQUE}
\ecoledoct{\og Physique en Île-de-France\fg}
\lab{DÉPARTEMENT DE PHYSIQUE DE L'ÉCOLE NORMALE SUPÉRIEURE \\ LABORATOIRE KASTLER BROSSEL}
\title{{Atomes de Rydberg piégés}}
\phdname{DOCTEUR DE L'ÉCOLE NORMALE SUPÉRIEURE}
\date{18}
\jury{
Dr. & Michel BRUNE & Directeur de thèse \\	
Dr. & Thierry LAHAYE & Rapporteur \\
Pr. & Shannon WHITLOCK & Rapporteur \\
Dr. & Bruno LABURTHE-TOLRA  & Examinateur    \\
Pr. & Jonathan HOME & Examinateur \\
Pr. & Agnès MAITRE & Examinateur
}
\titlefr{Piégeage laser d'atomes de Rydberg circulaires}
\resume{La simulation quantique offre un moyen très prometteur pour comprendre les systèmes quantiques corrélés macroscopiques. De nombreuses plateformes expérimentales sont en cours d'élaboration. Les atomes de Rydberg sont particulièrement intéressants grâce à leur forte interaction dipolaire de cours portée.
Dans notre manip, nous préparons et manipulons des ensembles d'atomes de Rydberg excités à partir d'un nuage atomique ultra-froid piégé magnétiquement sur une puce à atome supraconductrice. La dynamique de l'excitation est contrôlée par le processus d'excitation du laser. Le spectre d'énergie d'interaction atomique des $N$ corps est mesuré directment par spectroscopie micro-onde. Dans cette thèse, nous développons un modèle Monte Carlo rigoureux qui nous éclaire sur le processus d'excitation. En utilisant ce modèle, nous discutons de la possibilité de réaliser des simulations quantiques du transport d'énergie sur une chaîne 1D d'atomes de Rydberg de faible moment angulaire.
De plus, nous  proposons une plateforme innovante pour la réalisation de simulations quantiques. Elle repose sur une approche révolutionnaire basée sur un ensemble d'atomes de Rydberg dont le temps de vie est extrêmement long, qui interagissent fortement et qui sont piégés par laser. Nous présentons les résultats de simulations numériques et nous discutons du large éventail de problèmes qui peuvent être traités avec le modèle proposé.
}
\motscles{simulation quantique, atomes de Rydberg, atomes circulaires, interaction dipolaire, puce à atome, spectroscopie micro-ondre}


\titleen{Laser Trapping of Circular Rydberg Atoms}
\abstract{Quantum simulation offers a highly promising way to understand large correlated quantum systems, and many experimental platforms are now being developed. Rydberg atoms are especially appealing thanks to their strong and short-range dipole-dipole interaction.
	
	In our setup, we prepare and manipulate ensembles of Rydberg atoms excited from an ultracold atomic cloud magnetically trapped above a superconducting chip. The dynamics of the Rydberg excitation can be controlled through the laser excitation process. The many-body atomic interaction energy spectrum is then directly measured through microwave spectroscopy. This thesis develops a rigorous Monte Carlo model that provides an insight into the excitation process. Using this model, we discuss a possibility to explore quantum simulations of energy transport in a 1D chain of low angular momentum Rydberg atoms.
	
	Furthermore, we propose an innovative platform for quantum simulations. It relies on a groundbreaking approach, based on laser-trapped ensemble of extremely long-lived, strongly interacting circular Rydberg atoms. We present intensive numerical results as well as discuss a wide range of problems that can be addressed with the proposed model.}
\keywords{quantum simulation, Rydberg atoms, circular atoms, dipole-dipole interaction, atom chip, microwave spectroscopy}
