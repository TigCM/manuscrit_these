\chapter*{Introduction}\label{chapter:intro}
\addcontentsline{toc}{chapter}{Introduction} \mtcaddchapter
\markboth{Introduction}{Introduction}
\begin{fquote}[Michel Bakounine][Dieu et l'État]
La science comprend la pensée de la réalité, non la réalité elle-même, la pensée de la vie, non la vie. Voilà sa limite, la seule limite vraiment infranchissable pour elle, parce qu'elle est fondée sur la nature même de la pensée humaine, qui est l'unique organe de la science.
\end{fquote}

%\section*{Les merveilles du monde quantique, observées classiquement}
Une approche sommaire de la marche du savoir en sciences physiques se présente comme suit : nous observons le monde qui nous entoure et tentons d'en dégager des lois générales.
À partir de ces lois, nous prédisons de nouveaux phénomènes que nous cherchons à observer, dans la nature ou dans un laboratoire.
Observer ces phénomènes permet de raffiner les lois générales, de les confirmer ou de les infirmer.
Au fur et à mesure qu'avance notre compréhension du monde, les lois que nous formulons sont de plus en plus complexes.
\`A un certain point, les outils mathématiques développés et utilisés pour formuler ces lois deviennent abstraits, déconnectés de notre perception sensorielle du monde.
Quelle \og réalité \fg{} attribuer alors aux objets, aux grandeurs, aux fonctions qui représentent le monde ?

\newcommand{\rom}[1]{\uppercase\expandafter{\romannumeral #1\relax}}
Cette question, qui se pose dès les premières traces de l'investigation scientifique auxquelles l'histoire nous donne accès, prend une tournure dramatique avec la naissance de la théorie quantique au début du \rom{20}$^{\text{ième}}$ siècle.
Les débats font rage à propos de la dualité onde-particule, de la superposition d'états, de l'intrication quantique et de la réalité de la fonction d'onde.
Toutes les avancées de la science physique depuis lors ont confirmé la validité de la théorie quantique, depuis les fentes d'Young et les expériences de Stern et Gerlach jusqu'à l'utilisation de la précision des horloges atomiques pour le positionnement GPS, en passant par les expériences d'Aspect et le développement de la lumière laser.
Bien que les questions fondamentales de réalité des objets physiques et d'ontologie du monde quantique ne soient pas tranchées, la communauté internationale des physicien.ne.s s'accorde depuis plusieurs décennies sur le fait que l'on peut utiliser sans réserve la théorie quantique pour faire des prédictions phénoménologiques sur le monde microscopique.
Ces prédictions phénoménologiques sont toutes, \textit{in fine}, exprimées dans le cadre formel de la physique classique, le seul dans lequel nous puissions mesurer des grandeurs, extraire des informations d'un système sous observation.

\bigskip
%\section*{Les atomes de Rydberg et leurs interactions}

Les atomes de Rydberg, qui ont un électron de valence très éloigné du noyau, sont en quelque sorte des \og super-atomes \fg{}. Leur taille et leur sensibilité au rayonnement électromagnétique en font des outils de choix pour observer et sonder le monde quantique, avec des dimensions et grandeurs qui tendent à s'approcher de celles du monde de la physique classique.
En particulier, les atomes de Rydberg ont permis de réaliser dans le laboratoire des expériences de pensée que l'on croyait à jamais restreintes au monde de la spéculation intellectuelle.
De tels \textit{Gedankenexperimente} ont pu être mis \oe uvre, comme la manipulation d'états quantiques fondamentaux à un seul atome et un seul photon \cite{ENS_EXPTWOPHOTON},
l'observation directe de la quantification du champ électromagnétique dans une cavité \cite{ENS_QRABI},
le suivi en temps réel de la décohérence d'états non-classiques \cite{ENS_CAT,ENS_RMP},
la détection non-destructive de photons \cite{ENS_QND,ENS_COUNT05},
l'observation de sauts quantiques \cite{ENS_QNDZEROUN07,ENS_QNDCOLLAPSE07}
ou encore la reconstruction de la fonction de Wigner d'états non-classiques du champ \cite{ENS_FULLWIGNER08}.

Une autre particularité des atomes de Rydberg réside dans l'interaction entre eux, en tant que grands dipôles électriques.
Alors qu'il faut approcher deux atomes dans l'état fondamental à des distances de l'ordre de quelques nanomètres pour qu'ils interagissent, deux atomes de Rydberg interagissent à des distances de l'ordre de quelques micromètres.
Les interactions entre atomes de Rydberg sont déjà détectables, dans des gaz d'atomes chauds, par l'élargissement des spectres de transition optique entre l'état fondamental et les niveaux de Rydberg \cite{ENS_DENSEGAS,ENS_LASERSPEC5SP}.
Le domaine des atomes froids et ses impressionnantes avancées techniques visant à piéger et refroidir des nuages atomiques \cite{Pritchard1983,Dalibard1983,Dalibard1984,Chu1986,CCT1990} ouvre la voie à l'étude des interactions entre atomes de Rydberg dans des ensembles denses et froids.
Dans un cas limite idéal, le régime de \og gaz gelé \fg{} \cite{Gallagher1998,Pillet1998}, les atomes sont immobiles et les interactions au sein d'un ensemble d'atomes de Rydberg ont lieu avec des positions atomiques fixes.

\bigskip
%\section*{Notre étude des interactions}

L'objet d'étude premier du présent manuscrit est l'effet de ces interactions sur l'excitation et la dynamique d'un nuage dense d'atomes de Rydberg, dans le régime du \og gaz gelé \fg{}.
L'excitation résonante d'atomes de Rydberg au sein d'un gaz atomique ultra-froid subit le processus de \og blocage dipolaire \fg{} \cite{Lukin2001} :
l'interaction dipolaire décale la résonance des états qui contiendraient des niveaux de Rydberg spatialement proches et ces états ne sont donc pas accessibles par excitation laser.
L'excitation d'un premier atome de Rydberg bloque ainsi l'excitation d'autres atomes de Rydberg dans son voisinage.
L'exploitation de ce phénomène a été proposée pour réaliser des portes quantiques destinées à l'information quantique \cite{Lukin2001,Lukin2000b,Saffman2010}, ou encore pour réaliser une source déterministe d'atomes uniques \cite{Saffman2002}.

L'excitation désaccordée d'atomes de Rydberg au sein d'un gaz atomique ultra-froid subit un processus d'\og excitation facilitée \fg{} :
à une certaine distance d'un atome de Rydberg déjà excité, l'interaction dipolaire est égale au désaccord du laser et rend résonante l'excitation d'un deuxième atome de Rydberg.
RAUL fin p.4

Nous avons mené des expériences visant à mettre en évidence ces deux phénomènes complémentaires, en excitant des atomes de Rydberg à partir d'un nuage ultra-froid d'atomes de rubidium $\SI{87}{}$, piégés magnétiquement devant une puce atomique et refroidis près de la dégénérescence quantique.

\bigskip
\section*{La simulation quantique}

\bigskip
\section*{Le piégeage des atomes de Rydberg}

\section*{Problématisation}
\section*{Annonce du plan}
