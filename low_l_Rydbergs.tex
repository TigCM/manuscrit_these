\chapter{Interaction entre atomes de Rydberg sphériques et excitation de gaz dense}
\label{chapter:60s}
%Excitation optique d'atomes de Rydberg à bas $l$ et simulations

\section{Régimes d'excitation en nuage dense : blocage et facilitation}
	\subsection{Rappels sur l'interaction}
		\noindent hamiltonien d'interaction de paire \\
		approximation de $N$ atomes comme somme de $\frac{N(N-1)}{2}$ paires
	\subsection{mouvement des atomes au sein d'un gaz dense de Rydberg}
		\noindent  traitement classique du mouvement et ordres de grandeurs pertinents
	\subsection{Le blocage dipolaire et la facilitation}
		\noindent les deux régimes d'excitation déterminée par les interactions :\\
		explication du blocage dipolaire, et des effets qui le limitent (ailes de la gaussienne du nuage) \\
		pourquoi c'est difficile dans un BEC : mention du Pfau shift \\
		mention de la négligeabilité des excitations de paires ?

\section{Spectroscopie optique du nuage}
	\subsection{Description de la manip}
		\noindent spectres à différents temps d'interaction\\
		ou $N_rydberg$ en fonction du temps d'interaction pour différents detunings
		
	\subsection{Données : élargissement de la raie laser par interactions}
		\noindent conséquence de la facilitation
		
\section{Modèle de la dynamique d'excitation}
	\subsection{Simulations}
		\noindent modèle d'équation de taux\\
		\noindent résultats de simulations comparés aux manips\\
	\subsection{Les limites du modèle}
		%\noindent question du chauffage
		\noindent photons thermiques et apparition de niveaux $p$ \\
		LIRE T. PORTO
		
\section{Spectroscopie microonde du nuage : voir le mouvement}
	\subsection{Description de la manip}
		\noindent spectroscopie 60s-57s et son spectre d'excitation : comment cela nous donne accès au spectre des énergies d'interaction\\
		sonder le nuage à différents moments de son explosion
	\subsection{Données et accord avec les simulations}
		\noindent présenter les courbes de Raul.IV.3.2
		

\section*{Conclusion}
		\noindent il faut prendre en compte le mouvement, mais aussi les transferts thermiques vers les niveaux $p$
		
%\section{Spectroscopie microonde du nuage}
%	\subsection*{Spectre des énergies d'interaction du nuage}
%		\noindent détails sur la spectro 60s-57s, dont la quasi absence de terme d'échange dans l'interaction
%	\subsection*{Mouvement du nuage de Rydbergs}
%		\noindent Le gaz gelé ne marche pas !