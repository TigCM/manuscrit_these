\chapter{Des atomes froids en environnement cryogénique}
\label{chapter:setup_coldatoms}

\section{Le cryostat}
	\noindent description rapide du cryostat\\
	\noindent feedthrough pour les courants de bobines et de puce ?

\section{La puce et les bobines}
	\noindent design de la puce et un petit mot sur la fabrication \\
	\noindent bobines supras ?
	
\section{Séquence de piégeage et refroidissement}
	\subsection*{Piégeage magnéto-optique}
		\noindent 2d-mot, quad, u-mot, mélasse \\
	\subsection*{Piégeage magnétique}
		\noindent pompage optique et piège magnétique
	\subsection*{Refroidissement évaporatif jusqu'au BEC}
		\noindent dispositif de refroidissement RF

\section{Imagerie atomique}
	\subsection*{Optique d'imagerie}
		\noindent front et side
	\subsection*{L'imagerie par absoprtion}
		\noindent traitement d'image : absorption et absorption "nolog" \\
		\noindent mention de la réduction des franges ?
	
\section{Nuages typiques}
	\noindent qu'obtient-t-on comme MOTs, mélasses, nuages ultra-froids avec notre dispositif
	
%		
%\section{excitation et détection d'atomes de Rydberg, contrôle du champ électrique}
%	\subsection*{schémas d'excitation}
%		Lhomond et CdF
%		
%	\subsection*{schémas de détection}
%		state selective ionization
%		signaux d'ionisation et toutes les subtilités
%		
%	\subsection*{contrôle du champ lors de l'excitation ET de la détection}
%		Lhomond et CdF
%		
%\section{problème des champs électriques près d'une puce}
%
%\section{spectroscopie microonde}