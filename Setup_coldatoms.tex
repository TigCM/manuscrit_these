%\chapter{Des atomes froids en environnement cryogénique}
%\label{chapter:setup_coldatoms}
%
%\section{Le cryostat}
%	\noindent description rapide du cryostat\\
%	\noindent feedthrough pour les courants de bobines et de puce ?
%
%\section{La puce et les bobines}
%	\noindent design de la puce et un petit mot sur la fabrication \\
%	\noindent bobines supras ?
%	
%\section{Séquence de piégeage et refroidissement}
%	\subsection*{Piégeage magnéto-optique}
%		\noindent 2d-mot, quad, u-mot, mélasse \\
%	\subsection*{Piégeage magnétique}
%		\noindent pompage optique et piège magnétique
%	\subsection*{Refroidissement évaporatif jusqu'au BEC}
%		\noindent dispositif de refroidissement RF
%
%\section{Imagerie atomique}
%	\subsection*{Optique d'imagerie}
%		\noindent front et side
%	\subsection*{L'imagerie par absoprtion}
%		\noindent traitement d'image : absorption et absorption "nolog" \\
%		\noindent mention de la réduction des franges ?
%	
%\section{Nuages typiques}
%	\noindent qu'obtient-t-on comme MOTs, mélasses, nuages ultra-froids avec notre dispositif
	
%		
%\section{excitation et détection d'atomes de Rydberg, contrôle du champ électrique}
%	\subsection*{schémas d'excitation}
%		Lhomond et CdF
%		
%	\subsection*{schémas de détection}
%		state selective ionization
%		signaux d'ionisation et toutes les subtilités
%		
%	\subsection*{contrôle du champ lors de l'excitation ET de la détection}
%		Lhomond et CdF
%		
%\section{problème des champs électriques près d'une puce}
%
%\section{spectroscopie microonde}

\chapter{Des atomes de Rydberg froids en environnement cryogénique}
\label{chapter:setup_coldatoms_Rydberg}

\section{Les atomes froids}
	\subsection{le cryostat et la puce}
	\noindent schéma et description du cryostat \\
	\noindent schéma et description de la puce supra et des bobines supra
	\subsection{séquence de piégeage et refroidissement}
		\noindent piégeage magnéto-optique : 2D-MOT, QUAD-MOT, U-MOT \\
		
		\noindent piégeage magnétique de Ioffe Pritchard : principe et potentiel créé par le fil Z (cf. code Mathematica Radia)\\
		
		\noindent refroidissement évaporatif jusqu'au BEC : principe de l'évaporation RF et fils d'évap sur la puce
		
	\subsection{imagerie atomique}
		\noindent optique d'imagerie : schéma optique et caractéristiques des caméras \\
		
		\noindent imagerie par absorption : transition sonde, intensité de saturation \\
		\noindent imagerie par réflexion sur la puce : spécificités de la géométrie et double absoprtion du faisceau sonde \\
		
		\noindent tweaks : absorption no-log et réduction des franges : qu'a-t-on utilisé comme méthodes de traitement pour améliorer notre imagerie $\rightarrow$ un paragraphe sur la réduction de franges et un(ou deux) paragraphe(s) sur l'absorption no-log et sa pertinence dans les mesures de mélasses.
		
	\subsection{nuages typiques}
		\noindent quels MOTs, mélasses et nuages froids obtenus : tailles, températures, nombres d'atomes, distance à la puce.