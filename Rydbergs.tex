\chapter{Atomes de Rydberg alcalins en interaction}
\label{chapter:Rydberg}
%Atomes de Rydberg : à bas $l$ ou circulaires

Un atome de Rydberg est un atome dont un électron au moins occupe un état de grand nombre quantique principal $n$.
Par là, un atome de Rydberg présente des grandeurs et propriétés physiques surdimensionnées par rapport à un atome non excité ou peu excité.
On le remarque tout d'abord sur leur taille : un atome de rubidium dans le niveau $n=110$ a une extension spatiale de l'ordre d'$1 \mu m$, presque un million de fois plus grand que le rayon de Bohr qui représente l'ordre de grandeur caractéristique de la taille d'un atome dans son état fondamental.

\section{Théorie générale des Rydberg}
	\subsection{Hamiltonien de l'atome de Rydberg et défaut quantique}
%cf Raul I.1
		\noindent l'atome d'hydrogène : énergie des niveaux $E(n,l,j)= - \frac{E_I}{n^2}$
		avec $E_I = \frac{1}{1+\frac{m_e}{M}}\frac{m_e q^4}{32\pi^2 \epsilon _0^2 \hbar ^2}$ l'énergie d'ionisation de l'électron dans le niveau $1S$ de l'atome d'hydrogène.
		Etats propres : $\psi(r,\theta,\phi) = R_{nl}(r)\cdot Y_l^{m_l}(\theta,\phi)$
		\\
		+ structure fine $j,m_j$
		\bigskip
		
		\noindent le défaut quantique comme un $n$ effectif : 
		\[
		E(n,l,j) = \frac{E_I}{(n-\delta_{nlj})^2}
		\]
		avec $\delta(n,l,j)=\delta_{lj,0} + \frac{\delta_{lj,2}}{(n-\delta_{lj0})^2} + \frac{\delta_{lj,4}}{(n-\delta_{lj0})^4} + \frac{\delta_{lj,6}}{(n-\delta_{lj0})^6} + \cdots$ \\
		donner la table des défauts quantiques
		\bigskip
		
		\noindent mention de calcul du dipôle entre niveaux de Rydberg $\langle n,l,j,m_j | q \mathbf{r} |n',l',j',m_j'\rangle$
		
	\subsection{Niveaux à bas $l$}
	%	\noindent description rapide des spécificités et schéma de niveaux
		\noindent taille, dipole
		\noindent transitions vers niveaux proches, émission spontanée, temps de vie
	\subsection{Niveaux circulaires}
	%	\noindent description rapide des spécificités et schéma de niveaux
		\noindent taille, dipole
		\noindent transitions vers niveaux proches, émission spontanée, temps de vie
	%\subsection{Les grandes caractéristiques des Rydberg}
	%	\noindent gigantisme du dipole, sensibilité au champ EM, interactions
	%	\noindent lois d'échelle

\section{Atomes de Rydberg en interaction}
	\subsection{Deux atomes de Rydberg}
		\noindent hamiltonien d'interaction entre deux dipôles
		\[
		V_{dd} = \frac{1}{4\pi\epsilon_0 r^3} \left( \mathbf{d_1}\cdot \mathbf{d_2} - 3(\mathbf{d_1}\cdot \frac{\mathbf{r}}{r})(\mathbf{d_2}\cdot\frac{\mathbf{r}}{r}) \right)
		\]
		\noindent de l'interaction dipole-dipole générale au terme de Van der Waals en $1/r^6$ \\
		\noindent terme d'énergie et terme d'échange
	\subsection{les interactions entre Rydberg de bas $l$}
		\noindent origine du $C_6$ pour 60s-60s et $C_6/A_6$ avec les voisins\\
		reprendre Raul.figI.3 qui présente la partie radiale du dipôle 60s-ns en fonction de n\\
		\noindent principe du blocage dipolaire et facilitation (rapide)
	\subsection{les interactions entre Rydberg circulaires}
		\noindent $C_6$ pour 50c-50c et $C_6/A_6$ avec les voisins \\
		\noindent attention à l'anisotropie\\
		équivalent de la figure ci dessus (Raul.figI.3) pour les 50c, à modifier pour l'anisotropie
