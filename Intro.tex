\chapter*{Introduction}\label{chapter:intro}
\addcontentsline{toc}{chapter}{Introduction} \mtcaddchapter
\markboth{Introduction}{Introduction}
\begin{fquote}[Michel Bakounine][Dieu et l'État]
La science comprend la pensée de la réalité, non la réalité elle-même, la pensée de la vie, non la vie. Voilà sa limite, la seule limite vraiment infranchissable pour elle, parce qu'elle est fondée sur la nature même de la pensée humaine, qui est l'unique organe de la science.
\end{fquote}

%\section*{Les merveilles du monde quantique, observées classiquement}
%Une approche sommaire de la marche du savoir en sciences physiques se présente comme suit :
La marche du savoir en sciences physiques pourrait, de manière sommaire, se résumer ainsi :
nous observons le monde qui nous entoure et tentons d'en dégager des lois générales.
À partir de ces lois, nous prédisons de nouveaux phénomènes que nous cherchons à observer, dans la nature ou dans un laboratoire.
Observer ces phénomènes permet de raffiner les lois générales, de les confirmer ou de les infirmer.
Au fur et à mesure qu'avance notre compréhension du monde, les lois que nous formulons sont de plus en plus complexes.
\`A un certain point, les outils mathématiques développés et utilisés pour formuler ces lois deviennent abstraits, déconnectés de notre perception sensorielle du monde.
Quelle \og réalité \fg{} attribuer alors aux objets, aux grandeurs, aux fonctions qui représentent le monde ?

\newcommand{\rom}[1]{\uppercase\expandafter{\romannumeral #1\relax}}
Cette question, qui se pose dès les premières traces de l'investigation scientifique auxquelles l'histoire nous donne accès, prend une tournure dramatique avec la naissance de la théorie quantique au début du \rom{20}$^{\text{ième}}$ siècle.
Les débats font rage à propos de la dualité onde-particule, de la superposition d'états, de l'intrication quantique et de la réalité de la fonction d'onde.
Toutes les avancées de la science physique depuis lors ont confirmé la validité de la théorie quantique, depuis les fentes d'Young et les expériences de Stern et Gerlach jusqu'à l'utilisation de la précision des horloges atomiques pour le positionnement GPS, en passant par les expériences d'Aspect et le développement de la lumière laser.
Bien que les questions fondamentales de réalité des objets physiques et d'ontologie du monde quantique ne soient pas tranchées, la communauté internationale des physiciens et physiciennes s'accorde depuis plusieurs décennies sur le fait que l'on peut utiliser sans réserve la théorie quantique pour faire des prédictions phénoménologiques sur le monde microscopique.
Ces prédictions phénoménologiques sont toutes, \textit{in fine}, exprimées dans le cadre formel de la physique classique, le seul dans lequel nous puissions mesurer des grandeurs, extraire des informations d'un système sous observation.

\bigskip
%\section*{La physique quantique en laboratoire : métrologie, information quantique, questions fondamentales. Finir par la démonstration de propriétés fondamentales du monde}
Le domaine d'exploration de la physique quantique semble être intarissable.
La \og révolution quantique \fg{} du début du vingtième siècle a non seulement apporté une compréhension des phénomènes naturels du monde microscopique, elle a aussi permis l'avènement des avancées techniques et technologiques phares du siècle dernier grâce, entre autres, au développement des lasers et à la miniaturisation de l'électronique pour l'informatique.
Aujourd'hui, l'on peut dégager trois grands axes de recherche en physique quantique : la métrologie quantique, l'information quantique et l'investigation de systèmes élémentaires de plus en plus complexes.

La métrologie quantique consiste à mesurer des grandeurs physiques, telles que les constantes fondamentales, l'écoulement du temps, l'accélération de la gravité, avec la meilleure précision possible.
La stabilité et la sensibilité de ces objets, qu'on ose appeler \og élémentaires \fg{}, que sont les atomes en font d'excellentes sondes de l'environnement.
La mesure du temps par des techniques de métrologie quantique a atteint une précision telle que l'on sait se prémunir d'une déviation de l'ordre d'une seule seconde sur l'âge de l'Univers \cite{Ludlow2015,Oates2013}.

L'information quantique consiste à exploiter les propriétés particulières du monde quantique que sont la superposition d'états et l'intrication, à des fins de traitement de l'information au sens large.
Les efforts dans ce domaine sont symbolisés par la quête sans relâche de l'ordinateur quantique.
Un ordinateur quantique universel constitué de $N$ bits quantiques (ou \og qubits \fg{}) serait capable de traiter, en une seule fois, $2^N$ configurations différentes de ces $N$ qubits.
Cela représenterait une amélioration substantielle par rapport à un ordinateur classique, qui ne peut traiter qu'une seule configuration à la fois \cite{deutsch1985}.
Cette image est bien sûre naïve et doit être arraisonnée par les détails du traitement quantique de l'information mais elle donne une idée de l'intérêt qu'aurait la réalisation d'un ordinateur quantique universel performant.
Le développement d'un ordinateur quantique se heurte de plein fouet à la sensibilité des objets quantiques au couplage avec leur environnement, qui détruit rapidement les superpositions d'états et les propriétés d'intrication entre plusieurs particules.
Isoler de leur environnement un ensemble de quelques dizaines de particules quantiques pendant plusieurs secondes, tout en gardant un accès à ces particules afin de les manipuler, se révèle être un défi colossal \cite{QM_ZUREKDECOH91}.

Cependant, l'information quantique ne se limite pas au développement d'un ordinateur quantique universel.
Outre les techniques de cryptographie quantique, qui ont déjà fait la preuve de leur efficacité dans la sécurisation des communications, le domaine de l'information quantique s'étend à l'idée de la simulation quantique, sur laquelle nous reviendrons bientôt.
Celle-ci consiste, non pas à créer une plateforme universelle de calcul, mais à utiliser un système contrôlable et accessible comme modèle d'un système que l'on souhaite comprendre \cite{Feynman1982}.

Enfin, l'information quantique et les questions fondamentales sur les objets qui composent le monde microscopique se rejoignent dans les efforts théoriques et expérimentaux dévolus à l'étude raffinée des systèmes quantiques élémentaires.
Les travaux dans ce domaine, tels que le théorème de Bell et les expériences d'Aspect, nourrissent les réflexions concernant les fondements de la mécanique quantique \cite{QM_BELL64,QM_ASPECTGRANGIEREPR82,QM_ASPECTEPR82}.
La mesure et la manipulation de systèmes quantiques individuels ont représenté une avancée scientifique remarquable, soulignée par l'attribution du prix Nobel de physique 2012, permettant enfin de sonder directement les propriétés d'objets quantiques élémentaires et individuels.
Sur ces bases, la mesure et la manipulation de systèmes de plus en plus complexes, dans une approche ascendante de construction par l'agencement d'éléments simples, ouvre la voie vers la simulation quantique et vers l'étude de systèmes quantiques complexes.



\bigskip
%\section*{Les atomes de Rydberg et leurs interactions}
Les atomes de Rydberg, qui ont un électron de valence très éloigné du noyau, sont en quelque sorte des \og super-atomes \fg{}. Leur taille et leur sensibilité au rayonnement électromagnétique, associées à leur temps de vie très long par rapport aux niveaux faiblement excités, en font des outils de choix pour observer et sonder le monde quantique, avec des dimensions et grandeurs qui tendent à s'approcher de celles du monde de la physique classique.
En particulier, les atomes de Rydberg sont l'un des systèmes ayant permis de réaliser dans le laboratoire des expériences de pensée que l'on croyait à jamais restreintes au monde de la spéculation intellectuelle.
De tels \textit{Gedankenexperimente} ont pu être mis \oe uvre, comme la manipulation d'états quantiques fondamentaux à un seul atome et un seul photon \cite{ENS_EXPTWOPHOTON},
l'observation directe de la quantification du champ électromagnétique dans une cavité \cite{ENS_QRABI},
le suivi en temps réel de la décohérence d'états non-classiques \cite{ENS_CAT,ENS_RMP},
la détection non-destructive de photons \cite{ENS_QND,ENS_COUNT05},
l'observation de sauts quantiques \cite{ENS_QNDZEROUN07,ENS_QNDCOLLAPSE07}
ou encore la reconstruction de la fonction de Wigner d'états non-classiques du champ \cite{ENS_FULLWIGNER08}.

Une autre particularité des atomes de Rydberg réside dans l'interaction entre eux, en tant que grands dipôles électriques.
Alors qu'il faut approcher deux atomes dans l'état fondamental à des distances de l'ordre de quelques nanomètres pour qu'ils interagissent notablement, l'interaction entre deux atomes de Rydberg séparés de quelques micromètres est plus importante de plusieurs ordres de grandeurs.
Les interactions entre atomes de Rydberg sont déjà détectables, dans des gaz d'atomes chauds, par l'élargissement des spectres de transition optique entre l'état fondamental et les niveaux de Rydberg \cite{ENS_DENSEGAS,ENS_LASERSPEC5SP}.
Le domaine des atomes froids et ses impressionnantes avancées techniques visant à piéger et refroidir des nuages atomiques \cite{Pritchard1983,Dalibard1983,Dalibard1984,Chu1986,CCT1990} ouvre la voie à l'étude des interactions entre atomes de Rydberg dans des ensembles denses et froids.
Dans un cas limite idéal, le régime de \og gaz gelé \fg{} \cite{Gallagher1998,Pillet1998}, les atomes sont immobiles et les interactions au sein d'un ensemble d'atomes de Rydberg ont lieu avec des positions atomiques fixes.

%\bigskip
%\section*{Notre étude des interactions}

%L'objet d'étude premier du présent manuscrit est l'effet de ces interactions sur l'excitation et la dynamique d'un nuage dense d'atomes de Rydberg, dans le régime du \og gaz gelé \fg{}.
Dans ce régime de gaz gelé, l'excitation résonante d'atomes de Rydberg au sein d'un gaz atomique ultra-froid subit le processus de \og blocage dipolaire \fg{} \cite{Lukin2001} :
l'interaction dipolaire décale la résonance des états qui contiendraient des niveaux de Rydberg spatialement proches et ces états ne sont donc pas accessibles par excitation laser.
L'excitation d'un premier atome de Rydberg bloque ainsi l'excitation d'autres atomes de Rydberg dans son voisinage.
L'exploitation de ce phénomène a été proposée pour réaliser des portes quantiques destinées à l'information quantique \cite{Lukin2001,Lukin2000b,Saffman2010}, ou encore pour réaliser une source déterministe d'atomes uniques \cite{Saffman2002}.

L'excitation désaccordée d'atomes de Rydberg au sein d'un gaz atomique ultra-froid repose sur un processus d'\og excitation facilitée \fg{} :
à une certaine distance d'un atome de Rydberg déjà excité, l'interaction dipolaire est égale au désaccord du laser et rend résonante l'excitation d'un deuxième atome de Rydberg.
Ces calculs numériques sur de petits nuages, ou des simulations Monte Carlo de l'excitation dans de grands nuages, confirment ce phénomène de facilitation \cite{Robicheaux2005,Rost2007a,Evers2013,Cote2010}.
Celui-ci entraîne l'excitation d'agrégats d'atomes de Rydberg fortement corrélés spatialement, qui ont été observés expérimentalement \cite{Weidemueller2013,Pillet2012}.

Au-delà de leur effet sur l'excitation d'un ensemble d'atomes de Rydberg, les interactions dipolaires produisent aussi un effet mécanique qui peut être important.
En effet, les énergies d'interaction dipolaire au sein d'un ensemble en forte interaction peuvent être bien plus élevées que les énergies cinétiques initiales d'un gaz d'atomes froids.
Le mouvement relatif entre les atomes de Rydberg induit par ces interactions vient ainsi remettre en question le domaine de validité de l'approximation du gaz gelé.

L'objet d'étude premier du présent manuscrit est cet ensemble d'effets des interactions sur l'excitation et la dynamique d'un ensemble dense d'atomes de Rydberg.
Nous avons mené des expériences visant à mettre en évidence le blocage dipolaire, l'excitation facilitée et le mouvement relatif au sein d'un ensemble d'atomes de Rydberg, en excitant des atomes de Rydberg à partir d'un nuage ultra-froid d'atomes de rubidium $\SI{87}{}$, piégés magnétiquement devant une puce atomique et refroidis près de la dégénérescence quantique \cite{ENS_CHIPSPECTRO14,ENS_CHIPINTERACTION15}.
Les subtilités de l'excitation d'atomes de Rydberg près d'une puce sont exposées dans la thèse de Carla Hermann Avigliano \cite{PHD_HERMANN}.
Les expériences concernant l'influence des interactions sur l'excitation et la dynamique d'un nuage dense d'atomes de Rydberg sont présentées dans la thèse de Raul Celistrino Teixeira \cite{PHD_CELISTRINO}.
Afin d'éclairer ces expériences, nous avons développé un modèle numérique, fondé sur les techniques de simulation Monte Carlo, permettant de prédire l'excitation et la dynamique d'un nuage dense et froid d'atomes de Rydberg dont les développements successifs sont présentés dans les thèses de Raul Celistrino Teixeira \cite{PHD_CELISTRINO} et Thanh Long Nguyen \cite{PHD_NGUYEN}.
%Les évolutions successives de notre modèle sont présentées 
%L'accord qualitatif de ce modèle simple avec les données expérimentales 
%nous a poussés à envisager la conception d'une plateforme de simulation quantique à partir d'atomes de Rydberg.

\bigskip
%\section*{La simulation quantique}
Les fortes interactions entre atomes de Rydberg étant bien comprises et contrôlables, elles présentent un point de départ tout indiqué au développement d'un simulateur quantique.
L'idée de la simulation quantique, proposée par R. Feynman en 1982 \cite{Feynman1982}, est d'utiliser un système quantique expérimentalement contrôlable et accessible, noté $A$, comme modèle, comme simulateur, d'un système quantique expérimentalement difficile à contrôler et à mesurer, noté $B$.
Il s'agit alors d'imposer le hamiltonien du système $B$ au système $A$, de laisser ce dernier évoluer puis de mesurer ses propriétés, analogues à celles qu'aurait le système $B$ après la même évolution.

L'intérêt de la simulation quantique se trouve dans l'étude de problèmes quantiques complexes.
Les problèmes mettant en jeu $N$ corps en interaction en sont l'exemple parfait.
L'état d'un ensemble de $N$ spins $1/2$ évolue au sein d'un espace de Hilbert de dimension $2^N$.
Dans certains cas, calculer, même numériquement, l'évolution temporelle d'un tel système demande la réalisation d'opérations lourdes, telles que l'exponentiation, sur des matrices de $2^N \times 2^N$ éléments.
Dès lors que l'on dépasse un certain nombre $N$ de corps, de tels calculs ne sont plus à la portée des ordinateurs existant, même les plus puissants.
Le seuil de $N=\SI{40}{}$ spins $1/2$ est communément admis \cite{Cirac2003,Friedenauer2008,Lloyd1996,Raedt2007} comme limite au-delà de laquelle les méthodes numériques classiques ne fonctionnent plus.
Utiliser un simulateur quantique permettrait d'outrepasser ces limites et de résoudre ainsi des problèmes de matière condensée mettant en jeu un grand nombre de particules quantiques en interaction, comme les chaînes et réseaux de spins, ou comme les comportements de gaz d'électrons dans des solides.
La meilleure connaissance de ces systèmes pourrait alors déboucher sur la compréhension de mécanismes fondamentaux, tels que la supraconductivité à haute température, ou sur le développement de nouveaux matériaux.

Un simulateur quantique idéal présente les caractéristiques suivantes : il est sous contrôle expérimental total, l'on peut faire varier les paramètres décisifs de son hamiltonien et mesurer toutes ses observables.
De nombreux systèmes quantiques ont été proposés comme briques élémentaires d'un simulateur quantique, tels que les ions piégés \cite{Blatt2012,Schneider2012}, les atomes froids en réseau \cite{Jaksch2005,Lewenstein2007,Bloch2012,Bloch2008}, les circuits de qubits supra-conducteurs \cite{Houck2012}, les polaritons en cavité \cite{Tanese2014}, les réseaux de photons \cite{Aspuru-Guzik2012,Carusotto2013}, les molécules polaires \cite{Buechler2009}, les boîtes  quantiques \cite{Cai2013,Manousakis2002,Byrnes2007} et d'autres encore.

\bigskip
%\section*{Notre idée de simulateur quantique}

Plusieurs propositions de simulation quantique à partir d'atomes de Rydbderg on déjà été faites \cite{SchAPnleber2015,Lesanovsky2012,Dauphin2012,Hague2013}.
%TL-105à108}.
Les démonstrations expérimentales allant dans ce sens incluent entre autres l'observation du transport cohérent d'une excitation au sein d'une chaîne de trois atomes de Rydberg \cite{Barredo2015} et l'observation de la dynamique d'excitation vers les niveaux de Rydberg de réseaux d'une vingtaine d'atomes \cite{Labuhn2016}.
L'idée générale de la simulation quantique avec des atomes de Rydberg est de préparer une chaîne d'atomes de Rydberg en interaction, séparés de distances de l'ordre de quelques microns à quelques dizaines de micron.

Toutes ces propositions et démonstrations ont été faites avec des atomes de Rydberg de faible moment cinétique, ce qui leur impose des limitations particulières.
En effet, les forces mécaniques importantes induites par les interactions dipolaires modifient rapidement la géométrie d'un ensemble ordonné d'atomes de Rydberg, si ceux-ci ne sont pas piégés.
%Or le piégeage des atomes de Rydberg de faible moment cinétique se trouve présenter une certaine difficulté.

Plusieurs techniques sont envisageables pour piéger des atomes de Rydberg.
En premier lieu, il est possible de piéger les atomes de Rydberg dans un potentiel magnétique, comme l'on sait le faire pour les atomes dans l'état fondamental \cite{MX_RAITHELTRAPPERDYD05}.
Afin de réaliser une chaîne d'atomes de Rydberg séparés de quelques microns ou dizaines de microns de distance, il faudrait alors une chaîne de micro-pièges magnétiques.
De tels structures de piégeages peuvent être réalisées à l'aide de puces à atomes \cite{muller2010trapping}, mais il faut alors placer les atomes de Rydberg à proximité immédiate de telles puces, à des distances comparables avec l'espacement entre les atomes.
La difficulté de cette approche provient de l'extrême sensibilité des atomes de Rydberg au champ électromagnétique, dont le contrôle à proximité immédiate d'une surface est délicat.
%Les atomes de Rydberg étant extrêmement sensibles au champ électromagnétique, les bruits électriques et champs parasites à proximité d'une surface rendent irréaliste le contrôle fin des interactions dans de telles conditions.

Une alternative consiste à piéger des atomes dans l'état fondamental dans un réseau optique, puis à les habiller avec des niveaux de Rydberg \cite{Johnson2010,Zeiher2016,Bijnen2015,Glaetzle2015,Macri2014}
%TL-111-115}.
La réalisation d'un réseau optique en environnement cryogénique étant techniquement lourde, ces expériences sont réalisées à température ambiante, où le rayonnement du corps noir vient perturber les niveaux de Rydberg et les interactions entre eux.

Le piégeage laser des atomes de Rydberg, grâce au potentiel pondéro-moteur imposé à l'électron de valence par la lumière, a également été proposé \cite{MX_RAITHELTRAP00}.
Dans ce cas, les atomes de Rydberg de bas moment cinétique sont sujets à un processus de photo-ionisation qui réduit fortement leur temps de vie \cite{MX_RAITHELPHOTION13}.

\bigskip
%\section*{Notre idée de simulateur, les Rydberg circulaires et le piégeage des atomes de Rydberg}
Nous proposons une nouvelle approche, consistant à exciter des niveaux de Rydberg circulaires et à les piéger par laser dans un environnement cryogénique.
Une chaîne de tels atomes, en forte interaction dipôle-dipôle,  réalise un modèle expérimentalement accessible et contrôlable d'une chaîne de spins $1/2$.
Cette proposition est présentée dans la thèse de Thanh Long Nguyen \cite{PHD_NGUYEN} et développée plus en profondeur dans \cite{ENS_PRE_CIRCSIM}.

Les niveaux de Rydberg circulaires sont les niveaux de Rydberg de moment cinétique maximal.
Aux températures cryogéniques, ils présentent des temps de vie encore bien supérieurs, de plusieurs ordres de grandeur, à ceux des atomes de Rydberg de bas moment cinétique.
De plus, le processus de photo-ionisation qui limite la durée de vie des atomes de Rydberg de bas moment cinétique dans un piège pondéro-moteur laser devient négligeable pour les niveaux de Rydberg circulaires.
En plaçant les atomes de Rydberg circulaires dans un environnement permettant d'inhiber leur désexcitation par émission spontanée, nous pouvons espérer des durées de vie des niveaux de Rydberg de l'ordre de la minute.
Une chaîne d'une quarantaine d'atomes de Rydberg présenterait donc un temps de vie supérieur à la seconde.
Avec de telles durées de vie, il est envisageable de simuler l'évolution d'une chaîne de spins pendant $\SIrange{e4}{e5}{}$ temps caractéristiques d'interaction.

Une proposition de méthode pour préparer de façon déterministe une chaîne régulière et sans défaut d'une quarantaine d'atomes de Rydberg circulaires vient compléter notre proposition de simulateur quantique.
Cette méthode permet en outre la mesure, atome par atome, de n'importe quelle observable de spin.
Notre proposition remplit ainsi les critères d'un bon simulateur quantique que nous avons évoqués : contrôle expérimental du système et du hamiltonien, et mesure de toutes les observables.

Le chemin vers la réalisation de ce simulateur commence par l'excitation d'atomes de Rydberg circulaires sur une puce atomique, et se continuera par la démonstration de leur piégeage pondéro-moteur dans un faisceau laser.

\bigskip
%\section*{Annonce du plan}
Le présent manuscrit est divisé en cinq chapitres.
Dans le premier, nous rappelons les propriétés importantes des atomes de Rydberg individuels, vus comme des atomes hydrogénoïdes géants.
Les atomes de Rydberg présentent entre autres une très grande taille, de très grands moments dipolaires de transition entre niveaux atomiques et des temps de vie remarquables par rapport aux niveaux atomiques faiblement excités.
Puisque nous travaillons avec des atomes de rubidium, il nous est nécessaire d'introduire la théorie du défaut quantique permettant de rendre compte de leur différence avec le modèle de l'atome d'hydrogène.
Nous nous attardons ensuite sur les spécificités des atomes de Rydberg circulaires, qui ont un moment cinétique maximal.
Enfin, nous présentons la théorie de l'interaction dipôle-dipôle entre deux atomes de Rydberg.

Dans le deuxième chapitre, nous décrivons le dispositif expérimental sur lequel le travail de thèse ici présenté a été effectué.
Le premier aspect de ce dispositif consiste à piéger et refroidir des atomes de rubidium $87$ à proximité d'une puce atomique supraconductrice, dans un environnement cryogénique.
Les atomes de rubidium provenant d'une source lente externe au cryostat sont capturés devant la puce atomique par un piège magnéto-optique.
Ils sont ensuite transférés vers un piège magnétique et refroidis par évaporation jusqu'à la dégénérescence quantique.
Le second aspect de ce dispositif consiste à exciter, manipuler et détecter des atomes de Rydberg à partir du nuage d'atomes froids.
Les atomes de Rydberg sont excités par une transition laser à deux photons, puis manipulés grâce à des champs microondes classiques induisant des transitions entre niveaux de Rydberg voisins.
Ils sont ensuite détectés par ionisation en champ électrique, selon une méthode sélective permettant de distinguer entre les différents niveaux de Rydberg.
L'excitation et la manipulation d'atomes de Rydberg à proximité d'une surface présente certaines particularités que nous discutons.

Dans le troisième chapitre, nous présentons les expériences qui ont été menées pour mettre en évidence les effets de l'interaction entre atomes de Rydberg sur la dynamique d'excitation et sur le mouvement d'un ensemble dense et froid d'atomes de Rydberg.
Les spectres laser d'excitation depuis le niveau fondamental vers le niveau de Rydberg $\mathrm{60S}$ présentent un élargissement caractéristique dû aux phénomènes de blocage dipolaire et d'excitation facilitée.
La spectroscopie de la transition microonde entre le niveau $\mathrm{60S}$ et le niveau $\mathrm{57S}$ permet de sonder directement les énergies d'interaction au sein d'un nuage d'atomes de Rydberg en expansion mécanique.
Nous présentons l'évolution du modèle numérique que nous avons développé afin de reproduire les données expérimentales et de rendre compte des phénomènes dûs aux interactions, et discutons les limites de ce modèle.
Enfin, nous utilisons ce modèle, là où il est valide, afin de confirmer les propriétés géométriques des nuages d'atomes de Rydberg en interaction ainsi que la dynamique à court terme de leur excitation.

Dans le quatrième chapitre, nous présentons en détail notre proposition de simulateur quantique avec des atomes de Rydberg circulaires piégés.
Nous présentons d'abord le principe général du simulateur proposé : quel système est simulé par une chaîne d'atomes de Rydberg circulaires en interaction et sous quelles conditions, comment piéger et préserver pendant des temps très longs les atomes de Rydberg circulaires et enfin  comment préparer de façon déterministe une chaîne d'atomes de Rydberg circulaires et comment les détecter.
Dans un second temps, nous explorons plus en détail les limitations du temps de vie des niveaux de Rydberg circulaires, dues en grande partie au problème du mélange de niveaux.

Dans le cinquième et dernier chapitre, nous présentons nos dernières réalisations expérimentales : l'excitation d'atomes de Rydberg circulaires sur puce.
Nous décrivons pour cela le principe du passage adiabatique radio-fréquence, qui est la méthode utilisée pour exciter les niveaux de Rydberg circulaires, ainsi que les adaptations du dispositif expérimental qui ont été nécessaires à réaliser ce passage adiabatique.
Nous présentons ensuite les différents étapes de l'excitation des atomes depuis le niveau fondamental jusqu'au niveau de Rydberg circulaire $\mathrm{50C}$.
Enfin, l'outil de la spectroscopie microonde vers les niveaux de Rydberg voisins nous permet de commencer à caractériser les ensembles d'atomes de Rydberg circulaires excités à proximité de la puce atomique.
Nous concluons le présent manuscrit en évoquant les perspectives expérimentales à court terme en vue de la réalisation du simulateur quantique proposé.




